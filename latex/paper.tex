\documentclass{ifacconf}
\usepackage[english]{babel}
\usepackage[T1]{fontenc}
\usepackage[utf8]{inputenc}
\usepackage{natbib}

\bibpunct{(}{)}{;}{a}{,}{}

\begin{document}
\begin{frontmatter}
\title{Simulation of 2D physics of hand drawn objects using OpenCV and Box2D}
\author[Bratislava]{Michal Sedlák}
\address[Bratislava]{Faculty of Electrical Engineering and Information Technology, Slovak University of Technology,
Ilkovičova 3, 812 19 Bratislava, Slovakia
\\
(e-mail: michal.sedlak@stuba.sk)}
\begin{abstract}
%TODO Napisat abstrakt

\end{abstract}
\begin{keyword}
Maximum 5 keywords.
\end{keyword}
\end{frontmatter}

\section{Introduction}
This paper describes applying of Newtonian physics to hand drawn objects recognized in image from camera. Simulation of physics is used in many modern applications. You can find in implementations used by 3D drawing and animation programs, more complex used in game engines or exact and precise simulation in CAE programs. Paper describes process of animation of hand drawn object, from a capturing phase, over recognition of the objects, interpretation objects in physical engine, to animation of such objects. This approach can be applied in education of physics at elementary schools, in interactive blackboards, or in games.



Figure~\ref{fig:song} demonstrates placement of figures. Place figures at
the top or bottom of a column wherever possible, as close as possible
to the first references to the in the paper. Restrict them to
single-column width please.

\subsection{Equations}
Center equations and place the numbers on the right as follows:
\begin{equation}
a^2+b^2=c^2
\end{equation}

\begin{figure}
\vspace*{3cm}
\caption{Example of a figure, to demonstrate the effect of
  long captions which run onto a second line}
\label{fig:song}
\end{figure}

\section{Transfer of your files to PC'11}

If it is not possible to send the electronic version of the paper and
summary, the paper version (two originals, not a photocopy) of your
full paper and abstract should be sent to the conference secretariat
by registered mail or courier service. 

The electronic version of your full paper and abstract are to be sent
via web page of the conference. The registration process makes use of
an internet database system and consists of three consecutive steps
(detailed description and help will be available on-line):
\begin{enumerate}
\item Author registration to obtain author's PIN. Each author or
  participant of the conference will receive a unique number and
  password. This number will identify the author when submitting the
  paper, creating invoice, etc. Persons that participated at past PC have
  their PIN already registered.  Based on PIN and password, the
  authors are be able to update or modify their coordinates (address,
  email, affiliation, etc) at any time. This information is secured by
  a password that can be send to authors again if forgotten.
\item Paper registration. Each accepted paper with unique ID will be
  added to the database via internet form. This contains a field where
  files can be attached and it will be sent to organisers together
  with other information on the form. Only the respective authors will
  be able to upload the title, paper, and abstract. Also, they will be
  able to modify this information later.
\item Invoice generation. The participants of the conference will
  fill in the web form and an invoice will automatically be generated
  for them. 
\end{enumerate}

The name of your file can be arbitrary.

Since the final electronic version will be the PDF format we prefer to
receive PDF files. The authors are encouraged to use hypertext
capabilities of the Acrobat software.  Before sending the PDF files to
the please test viewing and printing your file with Adobe Acrobat
Reader (Version 5.0). If you send the PS file, test it with
Ghostview. 

Important note: Please do not generate PDF with version 1.5 or higher.
The Adobe company provides Reader for this PDF version only for some
computer platforms and some people may not be able to read your
documents. This is for example case of Acrobat 6. Check its settings
and choose optimisations for version 1.4.

If your file exceeds the size of 10 MB, contact the organisers as this
is the limit of the web server.

The list of references must be at the end of the paper. When
referring to them in the text, type the corresponding reference
as \cite{AbTaRu:54},  \cite{Abl:56},
\cite{Keo:58}.
References should be given in the standard style as below.

\section*{Acknowledgments}
The work was supported by a grant (No. NIL-I-007-d) from Iceland, Liechtenstein 
and Norway through the EEA Financial Mechanism and the Norwegian Financial 
Mechanism. This project is also co-financed from the state budget of the Slovak Republic..



%If you use BibTeX comment the second line, if not comment the first line
\bibliography{pcbib}
%\documentclass{ifacconf}
\usepackage[english]{babel}
\usepackage[T1]{fontenc}
\usepackage[utf8]{inputenc}
\usepackage{natbib}

\bibpunct{(}{)}{;}{a}{,}{}

\begin{document}
\begin{frontmatter}
\title{Simulation of 2D physics of hand drawn objects using OpenCV and Box2D}
\author[Bratislava]{Michal Sedlák}
\address[Bratislava]{Faculty of Electrical Engineering and Information Technology, Slovak University of Technology,
Ilkovičova 3, 812 19 Bratislava, Slovakia
\\
(e-mail: michal.sedlak@stuba.sk)}
\begin{abstract}
%TODO Napisat abstrakt

\end{abstract}
\begin{keyword}
Maximum 5 keywords.
\end{keyword}
\end{frontmatter}

\section{Introduction}
This paper describes applying of Newtonian physics to hand drawn objects recognized in image from camera. Simulation of physics is used in many modern applications. You can find in implementations used by 3D drawing and animation programs, more complex used in game engines or exact and precise simulation in CAE programs. Paper describes process of animation of hand drawn object, from a capturing phase, over recognition of the objects, interpretation objects in physical engine, to animation of such objects. This approach can be applied in education of physics at elementary schools, in interactive blackboards, or in games.



Figure~\ref{fig:song} demonstrates placement of figures. Place figures at
the top or bottom of a column wherever possible, as close as possible
to the first references to the in the paper. Restrict them to
single-column width please.

\subsection{Equations}
Center equations and place the numbers on the right as follows:
\begin{equation}
a^2+b^2=c^2
\end{equation}

\begin{figure}
\vspace*{3cm}
\caption{Example of a figure, to demonstrate the effect of
  long captions which run onto a second line}
\label{fig:song}
\end{figure}

\section{Transfer of your files to PC'11}

If it is not possible to send the electronic version of the paper and
summary, the paper version (two originals, not a photocopy) of your
full paper and abstract should be sent to the conference secretariat
by registered mail or courier service. 

The electronic version of your full paper and abstract are to be sent
via web page of the conference. The registration process makes use of
an internet database system and consists of three consecutive steps
(detailed description and help will be available on-line):
\begin{enumerate}
\item Author registration to obtain author's PIN. Each author or
  participant of the conference will receive a unique number and
  password. This number will identify the author when submitting the
  paper, creating invoice, etc. Persons that participated at past PC have
  their PIN already registered.  Based on PIN and password, the
  authors are be able to update or modify their coordinates (address,
  email, affiliation, etc) at any time. This information is secured by
  a password that can be send to authors again if forgotten.
\item Paper registration. Each accepted paper with unique ID will be
  added to the database via internet form. This contains a field where
  files can be attached and it will be sent to organisers together
  with other information on the form. Only the respective authors will
  be able to upload the title, paper, and abstract. Also, they will be
  able to modify this information later.
\item Invoice generation. The participants of the conference will
  fill in the web form and an invoice will automatically be generated
  for them. 
\end{enumerate}

The name of your file can be arbitrary.

Since the final electronic version will be the PDF format we prefer to
receive PDF files. The authors are encouraged to use hypertext
capabilities of the Acrobat software.  Before sending the PDF files to
the please test viewing and printing your file with Adobe Acrobat
Reader (Version 5.0). If you send the PS file, test it with
Ghostview. 

Important note: Please do not generate PDF with version 1.5 or higher.
The Adobe company provides Reader for this PDF version only for some
computer platforms and some people may not be able to read your
documents. This is for example case of Acrobat 6. Check its settings
and choose optimisations for version 1.4.

If your file exceeds the size of 10 MB, contact the organisers as this
is the limit of the web server.

The list of references must be at the end of the paper. When
referring to them in the text, type the corresponding reference
as \cite{AbTaRu:54},  \cite{Abl:56},
\cite{Keo:58}.
References should be given in the standard style as below.

\section*{Acknowledgments}
The work was supported by a grant (No. NIL-I-007-d) from Iceland, Liechtenstein 
and Norway through the EEA Financial Mechanism and the Norwegian Financial 
Mechanism. This project is also co-financed from the state budget of the Slovak Republic..



%If you use BibTeX comment the second line, if not comment the first line
\bibliography{pcbib}
%\documentclass{ifacconf}
\usepackage[english]{babel}
\usepackage[T1]{fontenc}
\usepackage[utf8]{inputenc}
\usepackage{natbib}

\bibpunct{(}{)}{;}{a}{,}{}

\begin{document}
\begin{frontmatter}
\title{Simulation of 2D physics of hand drawn objects using OpenCV and Box2D}
\author[Bratislava]{Michal Sedlák}
\address[Bratislava]{Faculty of Electrical Engineering and Information Technology, Slovak University of Technology,
Ilkovičova 3, 812 19 Bratislava, Slovakia
\\
(e-mail: michal.sedlak@stuba.sk)}
\begin{abstract}
%TODO Napisat abstrakt

\end{abstract}
\begin{keyword}
Maximum 5 keywords.
\end{keyword}
\end{frontmatter}

\section{Introduction}
This paper describes applying of Newtonian physics to hand drawn objects recognized in image from camera. Simulation of physics is used in many modern applications. You can find in implementations used by 3D drawing and animation programs, more complex used in game engines or exact and precise simulation in CAE programs. Paper describes process of animation of hand drawn object, from a capturing phase, over recognition of the objects, interpretation objects in physical engine, to animation of such objects. This approach can be applied in education of physics at elementary schools, in interactive blackboards, or in games.



Figure~\ref{fig:song} demonstrates placement of figures. Place figures at
the top or bottom of a column wherever possible, as close as possible
to the first references to the in the paper. Restrict them to
single-column width please.

\subsection{Equations}
Center equations and place the numbers on the right as follows:
\begin{equation}
a^2+b^2=c^2
\end{equation}

\begin{figure}
\vspace*{3cm}
\caption{Example of a figure, to demonstrate the effect of
  long captions which run onto a second line}
\label{fig:song}
\end{figure}

\section{Transfer of your files to PC'11}

If it is not possible to send the electronic version of the paper and
summary, the paper version (two originals, not a photocopy) of your
full paper and abstract should be sent to the conference secretariat
by registered mail or courier service. 

The electronic version of your full paper and abstract are to be sent
via web page of the conference. The registration process makes use of
an internet database system and consists of three consecutive steps
(detailed description and help will be available on-line):
\begin{enumerate}
\item Author registration to obtain author's PIN. Each author or
  participant of the conference will receive a unique number and
  password. This number will identify the author when submitting the
  paper, creating invoice, etc. Persons that participated at past PC have
  their PIN already registered.  Based on PIN and password, the
  authors are be able to update or modify their coordinates (address,
  email, affiliation, etc) at any time. This information is secured by
  a password that can be send to authors again if forgotten.
\item Paper registration. Each accepted paper with unique ID will be
  added to the database via internet form. This contains a field where
  files can be attached and it will be sent to organisers together
  with other information on the form. Only the respective authors will
  be able to upload the title, paper, and abstract. Also, they will be
  able to modify this information later.
\item Invoice generation. The participants of the conference will
  fill in the web form and an invoice will automatically be generated
  for them. 
\end{enumerate}

The name of your file can be arbitrary.

Since the final electronic version will be the PDF format we prefer to
receive PDF files. The authors are encouraged to use hypertext
capabilities of the Acrobat software.  Before sending the PDF files to
the please test viewing and printing your file with Adobe Acrobat
Reader (Version 5.0). If you send the PS file, test it with
Ghostview. 

Important note: Please do not generate PDF with version 1.5 or higher.
The Adobe company provides Reader for this PDF version only for some
computer platforms and some people may not be able to read your
documents. This is for example case of Acrobat 6. Check its settings
and choose optimisations for version 1.4.

If your file exceeds the size of 10 MB, contact the organisers as this
is the limit of the web server.

The list of references must be at the end of the paper. When
referring to them in the text, type the corresponding reference
as \cite{AbTaRu:54},  \cite{Abl:56},
\cite{Keo:58}.
References should be given in the standard style as below.

\section*{Acknowledgments}
The work was supported by a grant (No. NIL-I-007-d) from Iceland, Liechtenstein 
and Norway through the EEA Financial Mechanism and the Norwegian Financial 
Mechanism. This project is also co-financed from the state budget of the Slovak Republic..



%If you use BibTeX comment the second line, if not comment the first line
\bibliography{pcbib}
%\documentclass{ifacconf}
\usepackage[english]{babel}
\usepackage[T1]{fontenc}
\usepackage[utf8]{inputenc}
\usepackage{natbib}

\bibpunct{(}{)}{;}{a}{,}{}

\begin{document}
\begin{frontmatter}
\title{Simulation of 2D physics of hand drawn objects using OpenCV and Box2D}
\author[Bratislava]{Michal Sedlák}
\address[Bratislava]{Faculty of Electrical Engineering and Information Technology, Slovak University of Technology,
Ilkovičova 3, 812 19 Bratislava, Slovakia
\\
(e-mail: michal.sedlak@stuba.sk)}
\begin{abstract}
%TODO Napisat abstrakt

\end{abstract}
\begin{keyword}
Maximum 5 keywords.
\end{keyword}
\end{frontmatter}

\section{Introduction}
This paper describes applying of Newtonian physics to hand drawn objects recognized in image from camera. Simulation of physics is used in many modern applications. You can find in implementations used by 3D drawing and animation programs, more complex used in game engines or exact and precise simulation in CAE programs. Paper describes process of animation of hand drawn object, from a capturing phase, over recognition of the objects, interpretation objects in physical engine, to animation of such objects. This approach can be applied in education of physics at elementary schools, in interactive blackboards, or in games.



Figure~\ref{fig:song} demonstrates placement of figures. Place figures at
the top or bottom of a column wherever possible, as close as possible
to the first references to the in the paper. Restrict them to
single-column width please.

\subsection{Equations}
Center equations and place the numbers on the right as follows:
\begin{equation}
a^2+b^2=c^2
\end{equation}

\begin{figure}
\vspace*{3cm}
\caption{Example of a figure, to demonstrate the effect of
  long captions which run onto a second line}
\label{fig:song}
\end{figure}

\section{Transfer of your files to PC'11}

If it is not possible to send the electronic version of the paper and
summary, the paper version (two originals, not a photocopy) of your
full paper and abstract should be sent to the conference secretariat
by registered mail or courier service. 

The electronic version of your full paper and abstract are to be sent
via web page of the conference. The registration process makes use of
an internet database system and consists of three consecutive steps
(detailed description and help will be available on-line):
\begin{enumerate}
\item Author registration to obtain author's PIN. Each author or
  participant of the conference will receive a unique number and
  password. This number will identify the author when submitting the
  paper, creating invoice, etc. Persons that participated at past PC have
  their PIN already registered.  Based on PIN and password, the
  authors are be able to update or modify their coordinates (address,
  email, affiliation, etc) at any time. This information is secured by
  a password that can be send to authors again if forgotten.
\item Paper registration. Each accepted paper with unique ID will be
  added to the database via internet form. This contains a field where
  files can be attached and it will be sent to organisers together
  with other information on the form. Only the respective authors will
  be able to upload the title, paper, and abstract. Also, they will be
  able to modify this information later.
\item Invoice generation. The participants of the conference will
  fill in the web form and an invoice will automatically be generated
  for them. 
\end{enumerate}

The name of your file can be arbitrary.

Since the final electronic version will be the PDF format we prefer to
receive PDF files. The authors are encouraged to use hypertext
capabilities of the Acrobat software.  Before sending the PDF files to
the please test viewing and printing your file with Adobe Acrobat
Reader (Version 5.0). If you send the PS file, test it with
Ghostview. 

Important note: Please do not generate PDF with version 1.5 or higher.
The Adobe company provides Reader for this PDF version only for some
computer platforms and some people may not be able to read your
documents. This is for example case of Acrobat 6. Check its settings
and choose optimisations for version 1.4.

If your file exceeds the size of 10 MB, contact the organisers as this
is the limit of the web server.

The list of references must be at the end of the paper. When
referring to them in the text, type the corresponding reference
as \cite{AbTaRu:54},  \cite{Abl:56},
\cite{Keo:58}.
References should be given in the standard style as below.

\section*{Acknowledgments}
The work was supported by a grant (No. NIL-I-007-d) from Iceland, Liechtenstein 
and Norway through the EEA Financial Mechanism and the Norwegian Financial 
Mechanism. This project is also co-financed from the state budget of the Slovak Republic..



%If you use BibTeX comment the second line, if not comment the first line
\bibliography{pcbib}
%\input{paper.bbl}

\end{document}




\end{document}




\end{document}




\end{document}


